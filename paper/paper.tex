\documentclass[11pt]{article}
\usepackage[round]{natbib}
\usepackage[margin=1in]{geometry}
\geometry{letterpaper}
\usepackage{url}

\begin{document}
\title{PeerPerformance: Luck--Corrected Peer Performance Analysis in R}
\author{David Ardia\\
Institute of Financial Analysis, University of Neuch\^atel, Switzerland\\
D\'epartement de Finance, Assurance et Immobilier, Universit\'e Laval, Qu\'ebec, Canada
\and
Kris Boudt\\
Solvay Business School, Vrije Universiteit Brussel, Belgium\\
Faculty of Economics and Business, VU University Amsterdam, The Netherlands
}
%\date{}
	
\maketitle

\section*{Summary}

PeerPerformance is an R package \citep{R} for the peer--performance evaluation of financial investments with luck--correction. In particular, it implements the peer performance ratios of \citet{ArdiaBoudt2016} which measure the percentage of peers 
a focal (hedge) fund outperforms and underperforms, after correction
for luck. It is useful for fund or portfolio managers to benchmark their investments or screen a universe of new funds. In addition, the package implements the testing framework for the Sharpe and modified Sharpe ratios, described in \citet{LedoitWolf2008} and \citet{ArdiaBoudt2015}. The latest version of the package
is available at \url{https://github.com/ArdiaD/PeerPerformance}.

\bibliographystyle{plainnat}
\bibliography{paper}
	
\end{document}
